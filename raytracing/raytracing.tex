\chapter{Ray Tracing}
Die Idee beim \introduce{Ray Tracing} ist es, für jeden Pixel, alle Objekte zu finden, die diesen Pixel beeinflussen.
Anhand all dieser Objekte wird die Pixelfarbe bestimmt.
Dazu wird vom Rückwärtslichttransport ausgegangen.
Man startet an der Kamera und sucht alle Pfade, auf denen das Licht dort hin gelangt.
Dabei wird angenommen, dass der Lichttransport den Gesetzen der geometrischen Optik folgt.

\section{Abtastung}
Ein \introduce{Rasterbild} ist eine äquidistante Abtastung eines Bildsignals.
Das Bildsignal wird also vereinfachend als stückweise konstante Funktion aufgefasst.
Die bringt Probleme wie \introduce{Aliasing} oder den \introduce{Moiré-Effekt} mit sich.

\begin{Theorem}[\textsc{Nyquist}-\textsc{Shannon}-Abtasttheorem]
	Ein kontinuierliches, bandbegrenztes Signal mit einer maximalen Frequenz $f_{max}$ muss mit einer Frequenz echt größer als $2 f_{max}$ abgetastet werden, damit aus dem diskreten Signal das Ursprungssignal exakt rekostruiert werden kann.
\end{Theorem}

Ist die Abtastfrequenz zu gering, kommt es zu Aliasing.
Ein möglicher Lösungsansatz ist eine Vorfilterung des Signals, bei der hohe Frequenzen entfernt werden.
Dies ist jedoch im allgemeinen Fall nicht möglich.
Eine andere Möglichkeit ist eine \introduce{Überabtastung} mit anschließender Filterung.

\section{Lochkamera}
Am einfachsten zur Bildsynthese ist das Modell der Lochkamera.
Sie ist definiert durch die Position ihrer Öffnung und der Bildebene.
Da keine Linse verwendet wird, hat sie unbeschränkte Schärfentiefe.

Eine \introduce{virtuelle Kamera} ist definiert durch ihre Position und Blickrichtung, sowie die Orientierung der Vertikalen Achse.
Dazu die Breite und Höhe der Bildebene und ihr Abstand \emph{vor} der Kamera.

Bei der Bildsynthese kann \introduce{objektbasiert} oder \introduce{bildbasiert} vorgegangen werden.

Beim objektbasierten Rendern werden für jedes Objekt alle Pixel bestimmt, die es überdeckt.
Dann wird die Farbe dieser Pixel ermittelt.

Beim bildbasierten Rendern werden für jeden Pixel alle an dieser Stelle sichtbaren Objekte bestimmt.
Daraus wird die Pixelfarbe ermittelt.

\section{Ray Tracing}
Ray Tracing besteht aus drei Schritten, die in dieser Reihenfolge ausgeführt werden.
\begin{enumerate}
	\item \introduce{Ray Generation} Für jeden Pixel wird ein Strahl von der Kamera durch diesen Pixel erzeugt.
	\item \introduce{Ray Intersection} Für jeden Strahl wird das Objekt gefunden, das die Kamera an diesem Pixel sieht.
	Es ist das Objekt, das diesen Strahl schneidet und dessen Schnittpunkt am nächsten an der Kamera liegt.
	\item \introduce{Shading} Farbe und Schattierung dieses Objekts an dieser Stelle wird berechnet.
\end{enumerate}
