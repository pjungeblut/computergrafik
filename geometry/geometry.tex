\chapter{Analytische Geometrie}
Dieses Kapitel gibt einen sehr groben Überlick über einige Konzepte aus der anlytischen Geometrie, die in der Computergrafik wichtig sind.
Alles ist sehr informell und nur als Wiederholung bekannten Inhalts gedacht.

\section{Vektoren und Punkte}
Ein \introduce{Vektor} besteht aus einer Richtung und einer Länge.
Vektoren werden genutzt, um Verschiebungen oder Richtungen anzugeben.
Ein \introduce{Ortsvektor} ist der Vektor vom Ursprung zu einem Punkt $P$.

Vektoraddition und -skalierung funktioniert komponentenweise.
Die Länge eines Vektors ergibt sich durch $\norm{a} = \sqrt{\sum_{i = 1}^n a_i}$.
Vektoren der Länge $1$ heißen \introduce{Einheitsvektoren}.

\begin{Definition}[Skalarprodukt]
	Seien $a = \left(a_1 \ldots a_n\right)^T$ und $b = \left(b_1 \ldots b_n\right)^T$ Vektoren.
	$\dotproduct{a}{b} = a_1b_1 + \ldots + a_nb_n$ ist das \introduce{Skalarprodukt} von $a$ und $b$.
\end{Definition}

Das Skalarprodukt kann auch als Matrixmultiplikation $a^Tb$ aufgefasst werden.
Es gelten die folgenden Eigenschaften:
\begin{enumerate}[label=(\roman*)]
	\item $\dotproduct{a}{b} = \norm{a} \norm{b} \cdot \cos \phi$.
	Dabei ist $\phi$ der von $a$ und $b$ eingeschlossene Winkel.
	\item $\dotproduct{a}{a} = a^2$
	\item $\dotproduct{a}{b} = \dotproduct{b}{a}$ (\introduce{Symmetrie})
	\item $\dotproduct{\lambda a + b}{c} = \lambda \dotproduct{a}{c} + \dotproduct{b}{c}$ (\introduce{Linearität})
\end{enumerate}

\begin{Definition}[Kreuzprodukt]
	Seien $a = \left(a_1~a_2~a_3\right)^T$ und $b = \left(b_1~b_2~b_3\right)^T$ Vektoren.
	\[
		a \times b =
		\begin{pmatrix}
			a_1 \\
			a_2 \\
			a_3
		\end{pmatrix}
		\times
		\begin{pmatrix}
			b_1 \\
			b_2 \\
			b_3
		\end{pmatrix}
		=
		\begin{pmatrix}
			a_2b_3 - a_3b_2 \\
			a_3b_1 - a_1b_3 \\
			a_1b_2 - a_2b_1
		\end{pmatrix}
		=
		n \norm{a} \norm{b} \sin \phi
	\]
	heißt das \introduce{Kreuzprodukt} von $a$ und $b$.
	$\norm{n}$ ist $1$ und $n$ steht senkrecht zu der von $a$ und $b$ aufgespannten Ebene.
	$\phi$ ist wieder der von $a$ und $b$ eingeschlossene Winkel.
\end{Definition}
Für das Kreuzprodukt gelten die folgenden Identitäten:
\begin{enumerate}[label=(\roman*)]
	\item $\dotproduct{a}{a \times b} = \dotproduct{b}{a \times b} = 0$
	\item $\dotproduct{a}{b \times c} = \dotproduct{a \times b}{c}$
	\item $a \times (\lambda b + c) = \lambda(a \times b) + a \times c$
	\item $a \times (b \times c) = \dotproduct{\dotproduct{a}{c}}{b} - \dotproduct{\dotproduct{a}{b}}{c}$
	\item $\dotproduct{a \times b}{c \times d} = \dotproduct{a}{c}\dotproduct{b}{d} - \dotproduct{b}{c}\dotproduct{a}{d}$
\end{enumerate}

\begin{Beispiel}[Anwendung des Kreuzprodukts]
	Seien $a b, c$ die Eckpunkte eines Dreiecks und $n' = (b - a) \times (c - a)$.
	Dann ist $n = \frac{n'}{\norm{n'}}$ die Oberflächennormale und der $\frac{1}{2}\norm{n'}$ der orientierte Flächeninhalt des Dreiecks.
\end{Beispiel}

\section{Parameterdarstellungen}
Eine \introduce{Gerade} ist durch zwei Punkte $P \neq Q$ definiert.
\[
	g(t) = P + t \cdot (Q - P), \quad t \in \mathbb{R}
\]
Wählt man $t$ aus $\mathbb{R}_+$, erhält man eine Halbgerade von Punkt $P$ durch den Punkt $Q$.
Wird $t$ dagegen auf $[0, 1]$ eingeschränkt, erhält man die Strecke zwischen $P$ und $Q$.

Analog ist eine \introduce{Ebene} durch drei nicht kollineare Punkte $P, Q, R$ definiert.
\[
	g(s, t) = P + s \cdot (Q - P) + t \cdot (R - P), \quad s, t \in \mathbb{R}
\]
Die Vektoren $(Q - P)$ und $(R - P)$ spannen die Ebene auf.
Wählt man $s$ und $t$ aus $[0, 1]$, erhält man das Parallelogramm zwischen $(Q - P)$ und $(R - P)$.

\section{Koordinatensysteme}
Ein $n$-dimensionales \introduce{Koordinatensystem} wird durch eine Menge von $n$ linear unabhängigen Basisvektoren $B = \{b_1, \ldots, b_n\}$ definiert.
Die Basisvektoren müssen nicht gleich lang sein.
$B$ ist \introduce{orthogonal}, wenn alle Basisvektoren senkrecht zueinander stehen.
Haben zusätzlich alle Basisvektoren Länge $1$, heißt $B$ \introduce{orthonormal}.
Die Einheitsvektoren $\{e_1, \ldots, e_n\}$ bilden eine solche Orthonormalbasis.

\section{Baryzentrische Koordinaten}
\begin{Definition}
	Seien $P_1, \ldots, P_k$ Punkte des $\mathbb{R}^n$ und $k \leq n + 1$.
	Wenn ein Punkt $Q$ als Affinkombination ($\lambda_1 + \ldots \lambda_k = 1$)
	\[
		Q = \lambda_1 P_1 + \ldots + \lambda_k P_k
	\]
	geschrieben werden kann, so bezeichnet man $(\lambda_1, \ldots, \lambda_k)$ als \introduce{baryzentrische Koordinaten} von $Q$ bezüglich $P_1, \ldots, P_k$.
\end{Definition}

Alle Konvexkombinationen ($\lambda_1 P_1 + \ldots + \lambda_k P_k$) einer Punktmenge $P_1, \ldots, P_k$, bilden die \introduce{konvexe Hülle} der Punktmenge. 
Da ein Dreieck konvex ist, ist ein Punkt genau dann innerhalb des Dreiecks, wenn all seine baryzentrischen Koordinaten bezüglich der Eckpunkte des Dreiecks nicht negativ sind.

\begin{Beispiel}
	Sei $P_1, P_2, P_3$ ein Dreieck und $Q$ ein Punkt.
	Dann gilt für die baryzentrischen Koordinaten $\lambda_1, \lambda_2, \lambda_3$:
	\[
		\lambda_1 = \frac{\Delta(Q, P_2, P_3)}{\Delta(P_1, P_2, P_3)} \qquad
		\lambda_2 = \frac{\Delta(P_1, Q, P_3)}{\Delta(P_1, P_2, P_3)} \qquad
		\lambda_3 = \frac{\Delta(P_1, P_2, Q)}{\Delta(P_1, P_2, P_3)}
	\]
	Dabei ist $\Delta(A, B, C)$ der orientierte Flächeninhalt des Dreiecks $A, B, C$.
\end{Beispiel}
