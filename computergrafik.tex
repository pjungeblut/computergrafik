\documentclass{report}

% Font- und Inputencoding. Deutsche Sprache.
\usepackage[T1]{fontenc}
\usepackage[ngerman]{babel}
\usepackage[utf8]{inputenc}

% Abstand zwischen Punkten und Satzanfängen.
\frenchspacing

% Farben, vor allem ein Blau für Definitionen.
\usepackage{color}

% Einheiten.
\usepackage{siunitx}

% Mathe.
\usepackage{mathtools}

% Ein Befehl zur Einführen von Begriffen.
\newcommand{\introduce}[1]{\textcolor{blue}{#1}}

% Das d am Ende des Integranden.
\newcommand{\intd}{~\mathrm{d}}

% Ein kleinerer Abstand zwischen dem z und dem B bei z.B.
\newcommand{\zB}{z.\,B.~}

% Titel, Autor und Datum.
\title{KIT Computergrafik, WS 15/16}
\author{Paul Jungeblut}
\date{\today}

\begin{document}

% Eine Titelseite.
\maketitle
\newpage

% Eine Seite mit Inhaltverzeichnis und kleinem Disclaimer.
\tableofcontents
\vfill
Dieses Skript ist inoffiziell zur Vorlesung Computergrafik am KIT im Wintersemester 2015/2016 entstanden.
Dieses Skript erhebt keinen Anspruch auf Vollständigkeit und Korrektheit.

Die Vorlesung wurde von Prof. Dr. Carsten Dachsbacher gehalten und dieses Skript orientiert sich an seinen Folien.
\newpage

% Ein separates Dokument pro Kapitel.
\chapter{Bilder, Farbe und Perzeption}
In der Computergrafik geht es um die Erzeugung und Manipulation von Bildern.
Diese Bilder sind meiste 2D Arrays aus farbigen Pixeln.
Der Speicher, in dem die Farbe mit drei Werten für rot, grün und blau gespeichert wird, heißt \introduce{Framebuffer}.
Heutzutage sind 8 Bit pro Farbe in einem Framebuffer üblich.
Steht weniger zur Verfügung, müssen fehlende Farben durch Anordnung verfügbarer Farben nachgebildet werden.
Sind die Pixel ausreichend klein, werden so Mischfarben wahrgenommen.

\section{Transferfunktion}
Höhere RGB-Werte bedeuten eine hellere Farbe.
Wie hell genau eine Farbe erscheint, wird durch eine \introduce{Transferfunktion $f$} beschrieben.
Man kann die Transferfunktion für ein Graustufenbild oder per Farbkanal betrachten.
\[
	f : \left[0, N\right] \to \left[I_{min}, I_{max}\right]
\]
Dabei bildet $f$ vom Wert des Pixels auf eine Helligkeit zwischen der minimalen und maximalen Displayhelligkeit $I_{min}$ und $I_{max}$ ab.
Sie hängt von den physikalischen Eigenschaften des Displays ab.

Die maximale Helligkeit $I_{max}$ gibt an, wie hell ein Pixel sein kann.
Bei LCDs beträgt sie meist weniger als $10\%$ der Hintergrundbeleuchtung des Displays.
Die minimale Helliglkeit $I_{min}$ ist die Menge Licht, die für ein schwarzes Pixel emmitiert wird.

Neben dem vom Display emmitierten Licht, reflektiert auch noch Umgebungslicht mit Intensität $k$ an der Oberfläche.
Dieses hat einen großen Einfluss auf den Kontrast, der am Bildschirm wahrgenommen werden kann.
Der \introduce{Dynamikumfang}
\[
	R_d := \frac{I_{max} + k}{I_{min} + k}
\]
beschreibt den maximalen Kontrast des Displays.

Die Transferfunktion sollte so ausfallen, dass aufeinander folgende Pixelwerte keinen sichtbaren Helligkeitsunterschied haben.
Ist diese Forderung nicht erfüllt, können Bänder auf glatten Bildbereichen erscheinen.
Menschen können einen Helligkeitsunterschied von etwa 2\% wahrnehmen.
In dunklen Bereichen werden also kleinere Schritte der Transferfunktion benötigt.

\subsection{Mögliche Transferfunktionen}


\end{document}
