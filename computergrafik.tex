\documentclass{scrreprt}

% Font- und Inputencoding. Deutsche Sprache.
\usepackage[T1]{fontenc}
\usepackage[ngerman]{babel}
\usepackage[utf8]{inputenc}

% Typografisches.
% Abstand zwischen Punkten und Satzanfängen.
% Font-expansion für kleinere Wortzwischenräume beim Blocksatz.
\frenchspacing
\usepackage{microtype}

% Farben, vor allem ein Blau für Definitionen.
\usepackage{color}

% Einheiten.
\usepackage{siunitx}

% Bessere Aufzählungen mit eigenen Labels.
\usepackage{enumitem}

% Mathe.
% Umgebeungen werden gemeinsam und pro Kapitel nummeriert.
\usepackage{amsmath}
\usepackage{mathtools}
\usepackage[standard]{ntheorem}
\theoremstyle{break}\renewtheorem{Definition}{Definition}[chapter]
\theoremstyle{break}\renewtheorem{Theorem}[Definition]{Satz}
\theoremstyle{break}\renewtheorem{Beispiel}[Definition]{Beispiel}

% Ein Befehl zur Einführen von Begriffen.
\newcommand{\introduce}[1]{\textcolor{blue}{#1}}

% Das d am Ende des Integranden.
\newcommand{\intd}{~\mathrm{d}}

% Notation fürs Skalarprodukt und die Länge eines Vektors.
\newcommand{\dotproduct}[2]{\langle #1, #2 \rangle}
\newcommand{\norm}[1]{\vert #1 \vert}

% Ein kleinerer Abstand zwischen dem z und dem B bei z.B.
\newcommand{\zB}{z.\,B.~}

% Titel, Autor und Datum.
\title{KIT Computergrafik, WS 15/16}
\author{Paul Jungeblut}
\date{\today}

\begin{document}

% Römische Nummerierung vor dem eigentlichen Inhalt.
\pagenumbering{roman}

% Titelseite.
\maketitle
\newpage

% Inhaltverzeichnis und kleinem Disclaimer.
\tableofcontents
\vfill
Dieses Skript ist inoffiziell zur Vorlesung Computergrafik am KIT im Wintersemester 2015/2016 entstanden.
Es erhebt keinen Anspruch auf Vollständigkeit und Korrektheit.

Die Vorlesung wurde von Prof. Dr. Carsten Dachsbacher gehalten und das Skript orientiert sich stark an seinen Folien.
\newpage

% Arabische Nummerierung für den eigentlichen Inhalt. Start bei Seite 1.
\setcounter{page}{1}
\pagenumbering{arabic}

% Ein separates Dokument pro Kapitel.
\chapter{Bilder, Farbe und Perzeption}
In der Computergrafik geht es um die Erzeugung und Manipulation von Bildern.
Diese Bilder sind meiste 2D Arrays aus farbigen Pixeln.
Der Speicher, in dem die Farbe mit drei Werten für rot, grün und blau gespeichert wird, heißt \introduce{Framebuffer}.
Heutzutage sind 8 Bit pro Farbe in einem Framebuffer üblich.
Steht weniger zur Verfügung, müssen fehlende Farben durch Anordnung verfügbarer Farben nachgebildet werden.
Sind die Pixel ausreichend klein, werden so Mischfarben wahrgenommen.

\section{Transferfunktion}
Höhere RGB-Werte bedeuten eine hellere Farbe.
Wie hell genau eine Farbe erscheint, wird durch eine \introduce{Transferfunktion $f$} beschrieben.
Man kann die Transferfunktion für ein Graustufenbild oder per Farbkanal betrachten.
\[
	f : \left[0, N\right] \to \left[I_{min}, I_{max}\right]
\]
Dabei bildet $f$ vom Wert des Pixels auf eine Helligkeit zwischen der minimalen und maximalen Displayhelligkeit $I_{min}$ und $I_{max}$ ab.
Sie hängt von den physikalischen Eigenschaften des Displays ab.

Die maximale Helligkeit $I_{max}$ gibt an, wie hell ein Pixel sein kann.
Bei LCDs beträgt sie meist weniger als $10\%$ der Hintergrundbeleuchtung des Displays.
Die minimale Helliglkeit $I_{min}$ ist die Menge Licht, die für ein schwarzes Pixel emmitiert wird.

Neben dem vom Display emmitierten Licht, reflektiert auch noch Umgebungslicht mit Intensität $k$ an der Oberfläche.
Dieses hat einen großen Einfluss auf den Kontrast, der am Bildschirm wahrgenommen werden kann.
Der \introduce{Dynamikumfang}
\[
	R_d := \frac{I_{max} + k}{I_{min} + k}
\]
beschreibt den maximalen Kontrast des Displays.

Die Transferfunktion sollte so ausfallen, dass aufeinander folgende Pixelwerte keinen sichtbaren Helligkeitsunterschied haben.
Ist diese Forderung nicht erfüllt, können Bänder auf glatten Bildbereichen erscheinen.
Menschen können einen Helligkeitsunterschied von etwa 2\% wahrnehmen.
In dunklen Bereichen werden also kleinere Schritte der Transferfunktion benötigt.

\subsection{Quantisierung}
Die Transferfunktion kann verschieden quantisiert sein.
Die verschiedenen Möglichkeiten unterscheiden sich dabei in der größe der Helligkeitsschritte zwischen aufeinander folgenden Farbwerten.

Bei einer \introduce{linearen Quantisierung} (gleich große Helligkeitsschritte), muss jeder Schritt kleiner als 2\% von $I_{min}$ betragen.
Um Helligkeiten bis $I_{max}$ darzustellen werden
\[
	\frac{I_{max} - I_{min}}{0.02 \cdot I_{min}}
\]
Schritte benötigt.
Bei LCDs mit Dynamikumfang 100:1 sind dies etwa 5000 Schritte.
Dies würde 12-13 Bit pro Farbkanal erfordern.
Vorteil der linearen Quantisierung ist jedoch die einfache Arithmetik mit Pixelwerten.

Alternativ könnte die Transferfunktion exponentiell quantisiert sein, mit genau 2\% zwischen zwei Pixelwerten.
Bei einer \introduce{exponentiellen Quantisierung} ist $0 \mapsto I_{min}$, $1 \mapsto 1.02 \cdot I_{min}$, $2 \mapsto 1.02^2 \cdot I_{min}$, usw.
Da $\log_{10} 1.02 \approx \frac{1}{120}$, werden ca. 120 Schritte für eine Verzehnfachung der Helligkeit benötigt.
In diesem Fall reichen also 8 Bit gerade aus, um die 240 Schritte zu ermöglichen, die ein LCD mit Dynamikumfang 100:1 bräuchte.

Als Approximation der exponentiallen Quantisierung wird in der Praxis häufig eine \introduce{potenzfunktion basierte Quantisierung} eingesetzt.
\[
	I(n) = \left(\frac{n}{N}\right)^\gamma \cdot I_{max}
\]
Der Exponent $\gamma$ muss in diesem Fall immer mit angegeben werden.
Ist $\gamma = 1$ hat man eine lineare Quantisierung.

\subsection{$\gamma$-Korrektur}
In diesem Abschnitt gelten vereinfachend die Idealwerte $I_{min} = k = 0$ und $I_{max} = 1$.
Mit insgesamt $N$ Schritten ($N = 256$ bei $8$ Bit) wird ein Pixelwert $n$ auf die Intensität $I(n)$ abgebildet.
\[
	I(n) \propto \left(\frac{n}{N}\right)^\gamma
\]
Der $\gamma$-Wert charakterisiert das Display.
In der Computergrafik wird ein Pixelwert $\alpha$ aber üblicherweise in einem linearen Raum berechnet.
Bei der Darstellung will man, dass ein doppelter Wert doppelte Helligkeit bedeutet.
Pixelwerte werden daher \emph{direkt vor der Darstellung} einer \introduce{$\gamma$-Korrektur} unterzogen.
Damit $I(n) \propto \alpha$ ist, wird $\alpha \propto \alpha^{\frac{1}{\gamma}}$ verwendet.
Diese Korrektur wird unabhängig für jede Primärfarbe durchgeführt

\section{$\alpha$-Kanal}
Oft werden Bilder 32 Bit pro Pixel kodiert.
Ein Beispiel ist das RGBA-Format, wo neben den Primärfarben rot, grün und blau zusätzlich 8 Bit für einen \introduce{$\alpha$-Kanal} zur Verfügung stehen.
Der $\alpha$-Wert gibt die Opazität (Gegenteil von Transparenz) an.
Die Verwendung eines Alphakanals erlaubt es, Details von der Geometrie in die Textur zu verlagern und so das Rendern der Szene zu beschleunigen.

\section{Licht}
Licht ist elektoromagnetische Strahlung.
Eine elektormagnetische Welle hat eine Wellenlänge $\lambda$ und eine Frequenz $\nu = \frac{c}{\lambda}$.
Jede solche Wellenlänge repräsentiert eine \introduce{Spektralfarbe}.
Sichtbares Licht hat Wellenlängen zwischen \SI{380}{\nano\meter}-\SI{700}{\nano\meter}.
Licht ist in der Regel zusammengesetzt aus vielen verschiedenen Wellenlängen, jede mit einer bestimmten Intensität. $P(\lambda)$ ist die \introduce{Strahlungsleistung} der Wellenlänge $\lambda$.

Das menschliche Auge kann die spektrale Zusammensetzung von Licht nicht erfassen.
Es passt sich zudem den äußeren (physikalischen) Umständen an.
Als Rezeptoren dienen \introduce{Stäbchen} und \introduce{Zapfen}.
Die ca. 120 Millionen Stäbchen sind sehr lichtempfindlich und eignen sich für monochromatisches Nachtsehen.
Mit ca. 6-7 Millionen Zapfen ist trichromatisches Tagsehen möglich.
Es gibt drei Arten von Zapfen, die sich in ihrer Empfindlichkeit bezüglich verschiedener Lichtsprektren unterscheiden:
S-Zapfen (7\%) entsprechen dem blauen Licht, M-Zapfen (37\%)dem (gelb-)grünen und L-Zapfen (56\%) dem (orange-)roten Licht.

\subsection{Wahrnehmung von Licht}
Die Wahrnehmung von Licht erfolgt anhand des Tripels $(s,m,l)$ mit
\[
	s = \int S(\lambda)P(\lambda) \intd\lambda \text{,} \qquad
	m = \int M(\lambda)P(\lambda) \intd\lambda \text{,} \qquad
	l = \int L(\lambda)P(\lambda) \intd\lambda \text{.}
\]
Daraus folgt, dass es unterschiedliche Spektren mit unterschiedlichen Wellenlängen und Intensitäten gibt, die zur gleichen Wahrnehmung führen.
Diesen Effekt bezeichnet man als \introduce{Metamerismus}.
Der Metamerismus ist von elementarer Bedeutung in der Computergrafik, denn so kann ein Monitor mit drei Primärfarben den gleichen Eindruck vermitteln wie ein komplexes Spektrum.

\section{Farbräume}
Grundsätzlich kann zwischen \introduce{additiver} und \introduce{subtraktiver Farbmischung} unterschieden werden.

Bei additiver Farbmischung sind Rot, Grün und Blau die drei Primärfarben.
Die Farkombination entsteht durch \emph{Addition} der Spektren.

Bei der subtraktiven Farbmischung sind die Primärfarben Cyan, Gelb und Magenta.
Die Farbkombination entsteht durch \emph{Multiplikation} der Spektren.

Ein \introduce{Farbmodell} ist ein mathematisches Modell, in dem Farben durch Wertetupel beschrieben werden (\zB 3-Tupel bei RGB oder 4-Tupel bei CMYK).
Ein \introduce{Farbraum} ist die Menge aller Farben, die mit einem bestimmten Modell beschrieben werden können.
Die Tristimuluswerte beschreiben eine Farbe in einem bestimmten Farbraum eines Farbmodells.
Ohne Angabe des Farbmodells sind diese Werte nichtssagend.

\subsection{Graßmannsche Gesetze}
\begin{itemize}
	\item Farbe ist eine dreidimensionale Größe (\zB rot/grün/blau oder Farbton/Sättigung/Helligkeit).
	\item \introduce{Superpositionsprinzip}:
	Die Intensität einer additiv gemischten Farbe entspricht der Summe der Intensitäten der Ausgangsfarben.
	\item Der Farbton einer additiven Mischfarbe hängt nur vom Farbeindruck der Ausgangsfarben ab und nicht von deren Spektren.
	Auf die spektrale Zusammensetzung kann nicht rückgeschlossen werden.
	Beim Addieren von Spektren können einzelne Spektren also durch Metamere ersetzt werden, ohne dass sich der Farbeindruck ändert.
\end{itemize}

\subsection{RGB-Farbraum}
Im \introduce{RGB-Farbraum} dienen Rot, Grün und Blau als Primärfarben.
Die genaue Definition der Primärfarben häng vom jeweiligen RGB-Raum ab.
Es handelt sich um einen dreidimensionalen Farbraum mit
\[
	C = rR + gG + bB \in [0,1]^3 \text{.}
\]
Die Koeffizienten $r,g,b$ werden \introduce{Tristimuluswerte} genannt.
Zur Bestimmung Helligkeit kann die Luminanzapproximation
\[
	Y = 0.3r + 0.59g + 0.11b
\]

\subsection{CMY(K)-Farbraum}
Der \introduce{CMK}-Farbraum ist ein subtraktiver Farbraum mit den Primärfarben Cyan, Magenta und Gelb.
Er ist dual zum RGB-Farbraum:
\[
	\begin{pmatrix}
		C \\
		M \\
		Y
	\end{pmatrix}
	=
	\begin{pmatrix}
		1 \\
		1 \\
		1
	\end{pmatrix}
	 -
	\begin{pmatrix}
		R \\
		G \\
		B
	\end{pmatrix}
\]
verwendet werden.
Beim Drucken wird oft noch Schwarz als vierte Primärfarbe verwendet.
Man spricht dann vom \introduce{CMYK}-Farbraum.
\[
	K = \min\{C, M, Y\} \text{,} \qquad
	\begin{pmatrix}
		C' \\
		M' \\
		Y' \\
		K
	\end{pmatrix}
	=
	\begin{pmatrix}
		C - K \\
		M - K \\
		Y - K \\
		K
	\end{pmatrix}
\]

\subsection{HSV-Farbraum}
Der \introduce{HSV-Farbraum} besteht aus Farbton (engl. hue), Sättigung (engl. saturation) und Helligkeit (engl. value).
Er ist weder additiv noch subtraktiv aber recht intuitiv und findet deshalb oft Anwendung in Benutzerschnittstellen.

\subsection{CIE Color Matching Funktionen}
Eine \introduce{Color Matching Funktion} gibt an, wie die Primärfarben addiert werden müssen, um eine Spekralfarbe zu reprodizieren.
Nicht jede wahrnehmbare Farbe lässt sich durch Addition dreier Primärfarben darstellen.
In diesem Fall muss eine der Primärfarben zur Referenzfarbe addiert werden.
Mit den restlichen Primärfarben kann die Farbe dann nachgebildet werden.

Zur Berechnung einer metameren Farbe im selben RGB-Farbraum, müssen die Trsitimuluswerte $r, g, b$ der folgenden \introduce{Color Matching Funktionen} berechnet werden.
\[
	r = \int \bar{r}(\lambda)P(\lambda) \intd \lambda \text{,} \qquad
	g = \int \bar{g}(\lambda)P(\lambda) \intd \lambda \text{,} \qquad
	b = \int \bar{b}(\lambda)P(\lambda) \intd \lambda
\]
Dabei stellen $\bar{r}, \bar{g}, \bar{b}$ die Spektren der Primärfarben dar.
Dabei können wegen oben genanntem Effekt jedoch negative Vergleichswerte entstehen.
Die Konsequenz:
Einige Spektralfarben sind nicht realisierbar.
RGB ist also kein perfekter Farbraum, dafür jedoch realisierbar.

\subsection{Der XYZ-Farbraum}
Ziel des \introduce{XYZ-Farbraums} ist es, alle wahrnehmaren Farben beschreiben zu können.
Es ist demanch ein Farbraum mit rein positive Color Matching Funktionen.
Die $Y$-Komponente des XYZ-Farbraums entspricht der Luminanz.
Die Konvertierung zum RGB-Farbraum ist eine lineare Abbildung.
\[
	\begin{pmatrix}
		X \\
		Y \\
		Z
	\end{pmatrix}
	= M \cdot
	\begin{pmatrix}
		R \\
		G \\
		B
	\end{pmatrix}
	\text{,} \qquad
	\begin{pmatrix}
		R \\
		G \\
		B
	\end{pmatrix}
	= M^{-1} \cdot
	\begin{pmatrix}
		X \\
		Y \\
		Z
	\end{pmatrix}
	\text{,} \qquad
	M =
	\begin{pmatrix}
		0.49 & 0.31 & 0.20 \\
		0.18 & 0.81 & 0.01 \\
		0.00 & 0.01 & 0.99
	\end{pmatrix}
\]

\subsection{Der xyY-Farbraum}
Für alle $k > 0$ repräsentiert $k(X, Y, Z)$ die gleiche Farbe, nur mit unterschiedlicher Intensität.
Von daher können die Werte auf die $X + Y + Z = 1$ Ebene normalisiert werden.
Eine anschließende Projektion auf die XY-Ebene (z = 0 setzen) enthält nach wie vor alle Farbtöne und Sättigungen.
Es gilt
\[
	x = \frac{X}{X + Y + Z} \text{,} \qquad
	y = \frac{Y}{X + Y + Z} \text{,} \qquad
	z = \frac{Z}{X + Y + Z} = 1 - x - y \text{.}
\]
Im \introduce{xyY-Farbraum} wird so die Information in Helligkeit und Chromazität (Farbe) aufgeteilt.
Es müssen die Werte $x$ und $y$ sowie die Helligkeit $Y$ angegeben werden.
Der Wert $z$ kann wie oben gezeigt, durch $x$ und $y$ berechnet werden und muss nicht mit gespeichert werden.
$X$ und $Z$ können dann wie folgt aus $x$, $y$ und $Y$ berechnet werden:
\[
	X = \frac{Y}{y}x \text{,} \qquad
	Z = \frac{Y}{y}(1 - x - y)
\]

\section{Chromazitätsdiagramme}
Ein \introduce{Chromazitätsdiagramm} enthält alle sichtbaren Farben, dem \introduce{Gamut} der menschlichen Wahrnehmung.
Der \introduce{Weißpunkt} ($x = y = z = \frac{1}{3}$) entspricht dabei in etwa dem Sonnenlicht.
Die Spektralfarben befinden sich entlang der Randkurve und entsprechen dem monochromatischem Licht.
Die \introduce{Purpurlinie} ist die Menge von gesättigten Farben, die ein Mensch wahrnehmen kann.
Es handelt sich dabei aber nicht um Spektralfarben.

Farben auf der Strecke zwischen zwei Punkten können durch Addition der Farben an den Endpunkten der Strecke gebildet werden.
Die \introduce{reine Farbe $C_p$} zu einer Farbe $C$ findet man auf dem Rand durch Verlängern der Strecke vom Weißpunkt durch $C$.
Die \introduce{Komplementärfarbe $C_c$} liegt auf der Linie durch den Weißpunkt auf dem gegenüberliegenden Rand.

Alle Farben innerhalb eines Dreiecks lassen sich durch Addition der Farben an den Eckpunkten des Dreiecks bilden.
Die darstellbaren Farben eines Ausgabegeräts werden durch dessen \introduce{Gamut} beschrieben.
Es sind alle Farben innerhalb des von den Primärfarben aufgespannten Dreiecks (bzw. Polygons).

\introduce{Gamut-Mapping} ist eine Abbildung zwischen zwei Gamuts mit dem Ziel Farbverschiebungen gering zu halten.

\chapter{Analytische Geometrie}
Dieses Kapitel gibt einen sehr groben Überlick über einige Konzepte aus der anlytischen Geometrie, die in der Computergrafik wichtig sind.
Alles ist sehr informell und nur als Wiederholung bekannten Inhalts gedacht.

\section{Vektoren und Punkte}
Ein \introduce{Vektor} besteht aus einer Richtung und einer Länge.
Vektoren werden genutzt, um Verschiebungen oder Richtungen anzugeben.
Ein \introduce{Ortsvektor} ist der Vektor vom Ursprung zu einem Punkt $P$.

Vektoraddition und -skalierung funktioniert komponentenweise.
Die Länge eines Vektors ergibt sich durch $\norm{a} = \sqrt{\sum_{i = 1}^n a_i}$.
Vektoren der Länge $1$ heißen \introduce{Einheitsvektoren}.

\begin{Definition}[Skalarprodukt]
	Seien $a = \left(a_1 \ldots a_n\right)^T$ und $b = \left(b_1 \ldots b_n\right)^T$ Vektoren.
	$\dotproduct{a}{b} = a_1b_1 + \ldots + a_nb_n$ ist das \introduce{Skalarprodukt} von $a$ und $b$.
\end{Definition}

Das Skalarprodukt kann auch als Matrixmultiplikation $a^Tb$ aufgefasst werden.
Es gelten die folgenden Eigenschaften:
\begin{enumerate}[label=(\roman*)]
	\item $\dotproduct{a}{b} = \norm{a} \norm{b} \cdot \cos \phi$.
	Dabei ist $\phi$ der von $a$ und $b$ eingeschlossene Winkel.
	\item $\dotproduct{a}{a} = a^2$
	\item $\dotproduct{a}{b} = \dotproduct{b}{a}$ (\introduce{Symmetrie})
	\item $\dotproduct{\lambda a + b}{c} = \lambda \dotproduct{a}{c} + \dotproduct{b}{c}$ (\introduce{Linearität})
\end{enumerate}

\begin{Definition}[Kreuzprodukt]
	Seien $a = \left(a_1~a_2~a_3\right)^T$ und $b = \left(b_1~b_2~b_3\right)^T$ Vektoren.
	\[
		a \times b =
		\begin{pmatrix}
			a_1 \\
			a_2 \\
			a_3
		\end{pmatrix}
		\times
		\begin{pmatrix}
			b_1 \\
			b_2 \\
			b_3
		\end{pmatrix}
		=
		\begin{pmatrix}
			a_2b_3 - a_3b_2 \\
			a_3b_1 - a_1b_3 \\
			a_1b_2 - a_2b_1
		\end{pmatrix}
		=
		n \norm{a} \norm{b} \sin \phi
	\]
	heißt das \introduce{Kreuzprodukt} von $a$ und $b$.
	$\norm{n}$ ist $1$ und $n$ steht senkrecht zu der von $a$ und $b$ aufgespannten Ebene.
	$\phi$ ist wieder der von $a$ und $b$ eingeschlossene Winkel.
\end{Definition}
Für das Kreuzprodukt gelten die folgenden Identitäten:
\begin{enumerate}[label=(\roman*)]
	\item $\dotproduct{a}{a \times b} = \dotproduct{b}{a \times b} = 0$
	\item $\dotproduct{a}{b \times c} = \dotproduct{a \times b}{c}$
	\item $a \times (\lambda b + c) = \lambda(a \times b) + a \times c$
	\item $a \times (b \times c) = \dotproduct{\dotproduct{a}{c}}{b} - \dotproduct{\dotproduct{a}{b}}{c}$
	\item $\dotproduct{a \times b}{c \times d} = \dotproduct{a}{c}\dotproduct{b}{d} - \dotproduct{b}{c}\dotproduct{a}{d}$
\end{enumerate}

\begin{Beispiel}[Anwendung des Kreuzprodukts]
	Seien $a b, c$ die Eckpunkte eines Dreiecks und $n' = (b - a) \times (c - a)$.
	Dann ist $n = \frac{n'}{\norm{n'}}$ die Oberflächennormale und der $\frac{1}{2}\norm{n'}$ der orientierte Flächeninhalt des Dreiecks.
\end{Beispiel}

\section{Parameterdarstellungen}
Eine \introduce{Gerade} ist durch zwei Punkte $P \neq Q$ definiert.
\[
	g(t) = P + t \cdot (Q - P), \quad t \in \mathbb{R}
\]
Wählt man $t$ aus $\mathbb{R}_+$, erhält man eine Halbgerade von Punkt $P$ durch den Punkt $Q$.
Wird $t$ dagegen auf $[0, 1]$ eingeschränkt, erhält man die Strecke zwischen $P$ und $Q$.

Analog ist eine \introduce{Ebene} durch drei nicht kollineare Punkte $P, Q, R$ definiert.
\[
	g(s, t) = P + s \cdot (Q - P) + t \cdot (R - P), \quad s, t \in \mathbb{R}
\]
Die Vektoren $(Q - P)$ und $(R - P)$ spannen die Ebene auf.
Wählt man $s$ und $t$ aus $[0, 1]$, erhält man das Parallelogramm zwischen $(Q - P)$ und $(R - P)$.

\section{Koordinatensysteme}
Ein $n$-dimensionales \introduce{Koordinatensystem} wird durch eine Menge von $n$ linear unabhängigen Basisvektoren $B = \{b_1, \ldots, b_n\}$ definiert.
Die Basisvektoren müssen nicht gleich lang sein.
$B$ ist \introduce{orthogonal}, wenn alle Basisvektoren senkrecht zueinander stehen.
Haben zusätzlich alle Basisvektoren Länge $1$, heißt $B$ \introduce{orthonormal}.
Die Einheitsvektoren $\{e_1, \ldots, e_n\}$ bilden eine solche Orthonormalbasis.

\section{Baryzentrische Koordinaten}
\begin{Definition}
	Seien $P_1, \ldots, P_k$ Punkte des $\mathbb{R}^n$ und $k \leq n + 1$.
	Wenn ein Punkt $Q$ als Affinkombination ($\lambda_1 + \ldots \lambda_k = 1$)
	\[
		Q = \lambda_1 P_1 + \ldots + \lambda_k P_k
	\]
	geschrieben werden kann, so bezeichnet man $(\lambda_1, \ldots, \lambda_k)$ als \introduce{baryzentrische Koordinaten} von $Q$ bezüglich $P_1, \ldots, P_k$.
\end{Definition}

Alle Konvexkombinationen ($\lambda_1 P_1 + \ldots + \lambda_k P_k$) einer Punktmenge $P_1, \ldots, P_k$, bilden die \introduce{konvexe Hülle} der Punktmenge. 
Da ein Dreieck konvex ist, ist ein Punkt genau dann innerhalb des Dreiecks, wenn all seine baryzentrischen Koordinaten bezüglich der Eckpunkte des Dreiecks nicht negativ sind.

\begin{Beispiel}
	Sei $P_1, P_2, P_3$ ein Dreieck und $Q$ ein Punkt.
	Dann gilt für die baryzentrischen Koordinaten $\lambda_1, \lambda_2, \lambda_3$:
	\[
		\lambda_1 = \frac{\Delta(Q, P_2, P_3)}{\Delta(P_1, P_2, P_3)} \qquad
		\lambda_2 = \frac{\Delta(P_1, Q, P_3)}{\Delta(P_1, P_2, P_3)} \qquad
		\lambda_3 = \frac{\Delta(P_1, P_2, Q)}{\Delta(P_1, P_2, P_3)}
	\]
	Dabei ist $\Delta(A, B, C)$ der orientierte Flächeninhalt des Dreiecks $A, B, C$.
\end{Beispiel}

\chapter{Ray Tracing}
Die Idee beim \introduce{Ray Tracing} ist es, für jeden Pixel, alle Objekte zu finden, die diesen Pixel beeinflussen.
Anhand all dieser Objekte wird die Pixelfarbe bestimmt.
Dazu wird vom Rückwärtslichttransport ausgegangen.
Man startet an der Kamera und sucht alle Pfade, auf denen das Licht dort hin gelangt.
Dabei wird angenommen, dass der Lichttransport den Gesetzen der geometrischen Optik folgt.

\section{Abtastung}
Ein \introduce{Rasterbild} ist eine äquidistante Abtastung eines Bildsignals.
Das Bildsignal wird also vereinfachend als stückweise konstante Funktion aufgefasst.
Die bringt Probleme wie \introduce{Aliasing} oder den \introduce{Moiré-Effekt} mit sich.

\begin{Theorem}[\textsc{Nyquist}-\textsc{Shannon}-Abtasttheorem]
	Ein kontinuierliches, bandbegrenztes Signal mit einer maximalen Frequenz $f_{max}$ muss mit einer Frequenz echt größer als $2 f_{max}$ abgetastet werden, damit aus dem diskreten Signal das Ursprungssignal exakt rekostruiert werden kann.
\end{Theorem}

Ist die Abtastfrequenz zu gering, kommt es zu Aliasing.
Ein möglicher Lösungsansatz ist eine Vorfilterung des Signals, bei der hohe Frequenzen entfernt werden.
Dies ist jedoch im allgemeinen Fall nicht möglich.
Eine andere Möglichkeit ist eine \introduce{Überabtastung} mit anschließender Filterung.

\section{Lochkamera}
Am einfachsten zur Bildsynthese ist das Modell der Lochkamera.
Sie ist definiert durch die Position ihrer Öffnung und der Bildebene.
Da keine Linse verwendet wird, hat sie unbeschränkte Schärfentiefe.

Eine \introduce{virtuelle Kamera} ist definiert durch ihre Position und Blickrichtung, sowie die Orientierung der Vertikalen Achse.
Dazu die Breite und Höhe der Bildebene und ihr Abstand \emph{vor} der Kamera.

Bei der Bildsynthese kann \introduce{objektbasiert} oder \introduce{bildbasiert} vorgegangen werden.

Beim objektbasierten Rendern werden für jedes Objekt alle Pixel bestimmt, die es überdeckt.
Dann wird die Farbe dieser Pixel ermittelt.

Beim bildbasierten Rendern werden für jeden Pixel alle an dieser Stelle sichtbaren Objekte bestimmt.
Daraus wird die Pixelfarbe ermittelt.

\section{Ray Tracing}
Ray Tracing besteht aus drei Schritten, die in dieser Reihenfolge ausgeführt werden.
\begin{enumerate}
	\item \introduce{Ray Generation} Für jeden Pixel wird ein Strahl von der Kamera durch diesen Pixel erzeugt.
	\item \introduce{Ray Intersection} Für jeden Strahl wird das Objekt gefunden, das die Kamera an diesem Pixel sieht.
	Es ist das Objekt, das diesen Strahl schneidet und dessen Schnittpunkt am nächsten an der Kamera liegt.
	\item \introduce{Beleuchtungsberechnung} Farbe und Schattierung dieses Objekts an dieser Stelle wird berechnet.
	Dazu können rekursiv weitere Strahlen erzeugt werden, um \zB reflektierende Oberflächen darzustellen.
\end{enumerate}

\section{Ray Generation}
Die virtuelle Kamera ist definiert durch ihr Projektionszentrum $e$ (engl. eye) und einen $up$-Vektor mit $\norm{up} = 1$.
Sei $z$ der Zielpunkt eines Strahls.
Definiere dann
\[
	w = \frac{(e - z)}{\norm{e - z}} \text{,} \qquad
	u = up \times w \text {,} \qquad
	v = w \times u \text{.}
\]
Dabei ist $w$ die negative Blickrichtung.

Die Bildebene ist gegeben durch ihren Abstand $d$ zur Kamera, ihren linken und rechten Rand $l$ und $r$ sowie ihren oberen und unteren Rand $b$ und $t$.
Strahlen von $e$ aus zu einem Punkt $s$ auf der Bildebene sind nun beschrieben durch:
\[
	s = \lambda_1 u + \lambda_2 v - dw \qquad \lambda_1 \in [l, r] \quad \lambda_2 \in [b, t] 
\]
Typischerweise ist das Sichtfeld symmetrisch, es gilt also $l = -r$ und $t = -b$.
Das Verhältnis aus der Breite zur Höhe des Bildschirms heißt \introduce{Aspect Ratio}.

\section{Ray Intersection}
Geometrische Objekte können auf drei verschiedene Arten beschrieben werden:
\begin{itemize}
	\item \introduce{Parameterdarstellung}
	Einsetzen aller gültigen Parameterwerte liefert alle Punkte des Objekts.
	\item \introduce{Explizite Darstellung}
	Es ist eine Funktion gegeben, die an jeder Position beschreibt, ob das Objekt an dieser Position ist.
	\item \introduce{Implizite Darstellung}
	Alle Punkte des Objekts bilden die Lösungsmenge eines Systems von Gleichungen.
\end{itemize}

\subsection{Kugelschnitt}
Alle Punkte auf der Kugeloberfläche $K$ haben Abstand $r$ vom Mittelpunkt $c = (c_x, c_y, c_z)$.
Die implizite Darstellung der Kugel ist
\[
	K = \{(x, y, z) \mid \norm{(x - c_x, y - c_y, z - c_z)} = r\} \text{.}
\]
Sei $r(t) = e + td$ ein mit $t \in \mathbb{R}_+$ parametriesierter Strahl.
Für den Schnittpunkt aus Kugel und Strahl ergibt sich:
\begin{align*}
	0 &= \norm{r(t) - c}^2 - r^2 \\
		&= \norm{e + td - c}^2 - r^2 \\
		&= \dotproduct{e + td - c}{e + td - c} - r^2 \\
		&= \underbrace{\dotproduct{e - c}{e - c} - r^2}_c + \underbrace{2 \dotproduct{td}{e - c}}_{b \cdot t} + \underbrace{t^2 \dotproduct{d}{d}}_{a \cdot t^2}
\end{align*}
Mit der Mitternachtsformel
\[
	t_{1,2} = \frac{-b \pm \sqrt{b^2 - 4ac}}{2a}
\]
lassen sich nun die Parameter $t_1$ und $t_2$ bestimmen.
\[
	D = b^2 - 4ac
\]
heißt ist die Diskriminante.
Ist $D < 0$, gibt es keinen Schnittpunkt.
Ist sie gleich $0$, gibt es genau einen Schnittpunkt bei $r(t_1) = r(t_2)$.
Ist $D$ positiv, gibt es bei $r(t_1)$ und $r(t_2)$ jeweils einen Schnittpunkt.
Die Parameter $t_1$ und $t_2$ können kleiner als $0$ sein.
In diesem Fall liegt der Schnittpunkt hinter der Kamera und sollte nicht betrachtet werden.

\subsection{Ebenenschnitt}
Eine Ebene im $\mathbb{R}^3$ hat die implizite Darstellung
\[
	E = \{(x, y, z) \mid ax + by + cz + d = 0, \quad
	a, b, c, d \in \mathbb{R}, \quad
	a, b, c, \neq 0\} \text{.}
\]
Mit zwei nicht kollinearen Vektoren in der Ebene lässt sich die Normale $n$ berechnen.
Sei $r(t) = e + td$ ein Strahl mit $\norm{d} = 1$.
Dazu sei $\dotproduct{x}{n} - d = 0$ mit $\norm{n}$ die Ebene in Hesse-Normalform.
\begin{align*}
	0 &= \dotproduct{e + td}{n} - d \\
	  &= \dotproduct{e}{n} + t \dotproduct{d}{n} - d
\end{align*}
Damit folgt für den Parameter $t$:
\[
	t = \frac{d - \dotproduct{e}{n}}{\dotproduct{d}{n}}
\]
Fall $\dotproduct{d}{n} = 0$ gilt, sind Strahl und Ebene parallel und es exitiert kein Schnittpunkt.
Andernfalls schneiden sich Strahl und Ebene im Punkt $r(t)$.
Wenn $t < 0$ ist, liegt der Schnittpunkt hinter der Kamera und sollte ignoriert werden.

\subsection{Dreiecksschnitt}
Um einen Schintt zwischen einem Strahl und einem Dreieck zu berechnen, muss zuerst ein Schnittpunkt des Strahl mit der vom Dreieck aufgespannten Ebene gefunden werden.
Die Koordianten des Schnittpunktes können dann in baryzentrische Koordinaten überführt und auf Positivität überprüft werden.

\section{Beleuchtungsberechnung}
Beleuchtung ist essentiell für einen dreidimensionalen Eindruck.
Ein wichtiger Teil der Beleuchtungsberechungen ist die \introduce{Reflexion}.
Es gibt zwei Extreme.
Bei der \introduce{spekularen Reflexion} wird das Licht nur anhand eines Strahls reflektiert, wobei Einfallswinkel gleich Ausfallswinkel gilt.
Dagegen wird das Licht bei der \introduce{diffusen/lambterschen Reflexion} zu gleichen Teilen in alle Richtungen gestreut.

Im Folgenden wird nur Reflexion an der Oberfläche von Objekten behandelt.

\subsection{Bidirectional Reflectance Distribution Function - BRDF}
Eine \introduce{BRDF (Bidirectional Reflectance Distribution Function)} ist ein radiometrisches Konzept, um die Reflexion an einem Oberflächenpunkt zu beschreiben.
Sie gibt das Verhältnis von ausgehendem zu einfallendem Licht an einem Oberflächenpunkt an.
Um Materialien abzubilden, muss die BRDF erst aufwendig an diesem Material gemessen werden.

\subsection{Phong-Beleuchtungsmodell}
Das \introduce{Phong-Beleuchtungsmodell} ist ein phänomenologisches (also physikalische nicht korrektes) Modell zur Darstellung der Reflexion, anhand von drei Komponenten, die von den Materialparametern $k_a$, $k_d$ und $k_s$ sowie dem Phong-Exponenten $n$ abhängen:
\begin{itemize}
	\item \introduce{Ambient}
	Die indirekte Beleuchtung, also Licht, das von anderen Oberflächen reflektiert wird.
	Es ergibt sich der Anteil $I_a = k_a \cdot I_L$.
	\item \introduce{Diffus}
	Der Anteil der lambertschen Reflexion.
	Für den diffusen Anteil ergibt sich $I_d = k_d \cdot I_L \cdot \cos \theta = k_d \cdot I_L \cdot \dotproduct{N}{L}$.
	Dabei ist $I_L$ die Intensität der Lichtquelle, $N$ die normierte Normale am Oberflächenpunkt und $L$ der normierte Vektor zur Lichtquelle.
	\item \introduce{Spekular}
	Spekulare Reflexion bzw. perfekte Spiegelung.
	Die spekulare Reflexion findet ausschließlich in Richtung $R_L$ statt.
	Der Vektor $R_L$ ist die Spiegelung des Vektors $L$ zur Lichtquelle an der Oberflächennormalen $N$.
	Sind alle Vektoren normiert, ergibt sich $R_L = 2N \cdot \dotproduct{N}{L} - L$.

	Durch gerichtete Reflexion entstehen Glanzlichter.
	Die Stärke der Spiegelung fällt für von $R_L$ verschiedene Richtungen stark ab.
	Der Abfall wird durch $\cos^n \alpha$ modelliert.
	Der spekulare Anteil ergibt sich damit zu $I_s = k_s \cdot I_L \cdot \cos^n \alpha = k_s \cdot I_L \cdot \dotproduct{R_L}{V}^n$.
\end{itemize}
Die Gesamtbeleuchtung ergibt sich durch
\begin{align*}
	I &= I_a + I_d + I_s \\
	  &= k_a \cdot I_L +
	  	 k_d \cdot I_L \cdot \dotproduct{N}{L} +
	  	 k_s \cdot I_L \cdot \dotproduct{R_L}{V}^n \text{.}
	\end{align*}
Die Reflexionskoeffizienten $k_a$, $k_d$ und $k_s$ sind theoretisch Wellenlängenabhängig und werden deshalb oft für drei Wellenlängen (rot, grün, blau) angegeben.

Diffuse Reflexionen haben meist die Farbe der Oberfläche.
Spekulare Reflexionen haben meist die Farbe der Oberfläche, wenn es sich um Metalle handelt.
Ansonsten oft die Farbe der Lichtquelle.

Bei der Berechnung der Beleuchtungen ist man nur an den Richtungen interessiert, für die die Skalarprodukte positiv sind.

Optional kann das Phong-Beleuchtungsmodell um einen Emmisionsterm ersetzt werden.

\subsection{Berechnung der Normalen}

\chapter{Transformationen und homogene Koordinaten}

\section{Transformationsgruppen}
\begin{itemize}
  \item \introduce{Euklidische Transformationen} erhalten Abstände und Inhaltsgrößen (Flächeninhalt, Volumen).
  Zu ihnen zählen die Translation, die Identität sowie die Rotation.
  \item \introduce{Ähnlichkeitsabbildungen} erhalten Winkel.
  Alle euklidischen Transformationen sind Ähnlichkeitsabbildungen.
  Außerdem gehören isotrope Skalierungen in diese Klasse.
  \item \introduce{Lineare Abbildungen} enthalten alle Skalierungen, Rotationen,  Spiegelungen und Scherungen.
  Sie sind \introduce{additiv}, $T(p + q) = T(p) + T(q)$, und \introduce{homogen}, $T(\lambda p) = \lambda T(p)$.
  \item \introduce{Affine Abbildungen} enthalten die linearen Abbildungen sowie Translationen.
  Parallele Linien bleiben bei affinen Abbildungen erhalten.
  Weiter sind affine Abbildungen \introduce{teilverhältnistreu}.
  \introduce{Projektive Abbildungen} enthalten alle affinen Abbildungen.
  Die einzige Forderung ist jedoch, dass Geraden auf Geraden abgebildet werden.
\end{itemize}

\section{2D Transformationen}

Transformationen lassen sich durch Matrizen darstellen.
Dies erlaubt eine leichte Hintereinanderausführung.
Dazu müssen nur die einzelnen Transformationsmatrizen multipliziert werden.

\subsection{Skalierung}
Eine \introduce{Skalierung} ändert Längen und Winkel (nur bei nicht isotropen Skalierungen, $s_x \neq s_y$).
\[
  \mathrm{scale}(s_x, s_y) =
  \begin{pmatrix}
    s_x & 0 \\
    0   & s_y
  \end{pmatrix}
\]

\subsection{Scherung (Transvektion)}
Unter einer \introduce{Scherung} versteht man die Verschiebung entlang einer Achse.
Flächeninhalte bleiben daher erhalten.
Es kann entlang beliebiger Achsen geschert werden.
\[
  \mathrm{shear}_x(s) =
  \begin{pmatrix}
    1 & s \\
    0 & 1
  \end{pmatrix}
  \qquad
  \mathrm{shear}_y(s) =
  \begin{pmatrix}
    1 & 0 \\
    s & 1
  \end{pmatrix}
\]

\subsection{Spiegelung}
\introduce{Spiegelungen} an einer Koordinatenachse sind negative Skalierungen.
Es kann entlang beliebiger Achsen gespiegelt werden.
\[
  \mathrm{reflect}_x =
  \begin{pmatrix}
    -1 & 0 \\
    0  & 1
  \end{pmatrix}
  \qquad
  \mathrm{reflect}_y =
  \begin{pmatrix}
    1 & 0 \\
    0 & -1
  \end{pmatrix}
\]

\subsection{Rotation}
Eine \introduce{Rotation} beschreibt eine Drehung um einen Winkel $\phi$.
\[
  \mathrm{rotate}(\phi)
  \begin{pmatrix}
    \cos \phi & -\sin \phi \\
    \sin \phi & \cos \phi
  \end{pmatrix}
\]

\section{3D Transformationen}

\subsection{Rotation}
Die folgenden drei Matrizen drehen um die x-, y- und z-Achse.
\begin{align*}
  \mathrm{R}_x(\phi) &=
  \begin{pmatrix}
    1 & 0         & 0 \\
    0 & \cos \phi & -\sin \phi \\
    0 & \sin \phi & \cos \phi
  \end{pmatrix}
  \qquad
  \mathrm{R}_y(\phi) =
  \begin{pmatrix}
    \cos \phi  & 0 & \sin \phi \\
    0          & 1 & 0 \\
    -\sin \phi & 0 & \cos \phi
  \end{pmatrix} \\
  \mathrm{R}_z(\phi) &=
  \begin{pmatrix}
    \cos \phi  & -\sin \phi & 0 \\
    \sin \phi  & \cos \phi  & 0 \\
    0          & 0          & 1
  \end{pmatrix}
\end{align*}
Allgemein werden Rotationen durch \introduce{orthogonale Matrizen} beschrieben.
Sie sind orientierungserhaltend und ihre Zeilen- bzw. Spaltenvektoren sind paarweise orthonormal.
Eine quadratische, reele Matrix $M$ ist orhtogonal, wenn $M^T \cdot M = M \cdot M^T = I$ gilt.
Es gilt also $M^{-1} = M^T$.

Seien $u$, $v$ und $w$ die Basisvektoren eines orthonormalen Koordinatensystems.
Die Rotationsmatrix
\[
  R_{uvw} = 
  \begin{pmatrix}
    u_x & u_y & u_z \\
    v_x & v_y & v_z \\
    w_x & w_y & w_u
  \end{pmatrix}
\]
bildet die Basisvektoren $u$, $v$ und $w$ auf die kartesischen Achsen ab.
Umgekehrt bildet $R_{uvw}^T$ die kartesischen Einheitsvektoren auf $u$, $v$ und $w$ ab.

Neben den orthogonalen Matrizen können Rotationen auch über sog. \introduce{Euler-Rotationen} berechnet werden.
Dabei wird nacheinander um verschiedene Koordinatenachsen rotiert.
Es gibt zum Beispiel die folgenden drei Möglichkeiten:
\[
  R_z \to R_x \to R_z \qquad
  R_z \to R_y \to R_z \qquad
  R_x \to R_y \to R_z
\]
Sind die Rotationswinkel um die x-, y- und z-Achse $\psi$, $\theta$ und $\phi$, ergibt sich die Rotationsmatrix
\begin{align*}
  R &= R_z(\phi) \cdot R_y(\theta) \cdot R_x(\psi) \\
  &=
  \begin{pmatrix}
    \cos\theta\cos\phi &
    \sin\psi\sin\theta\cos\phi-\cos\psi\sin\phi &
    \cos\psi\sin\theta\cos\phi+\sin\psi\sin\phi \\
    \cos\theta\sin\phi &
    \sin\psi\sin\theta\sin\phi+\cos\psi\cos\phi &
    \cos\psi\sin\theta\sin\phi-\sin\psi\cos\phi \\
    -\sin\theta &
    \sin\psi\cos\theta &
    \cos\psi\cos\theta
  \end{pmatrix}
  \text{.}
\end{align*}
Die Winkel $\psi$, $\theta$ und $\phi$ heißen \introduce{Euler-Winkel} und beschreiben zusammen mit Festlegung der Achsen und der Reihenfolge die Orientierung eines Objekts.

Oft will man um eine Achse $d$ und einen Winkel $\phi$ rotieren.
Dazu wird ein Orthonormalsystem mit $d$ als eine der Achsen benötigt.
Sei $d = \left(d_x, d_y, d_z\right)$ mit $\norm{d} = 1$.
Wähle nun \zB
\[
  e = \frac{1}{\sqrt{d_y^2+ d_z^2}}\left(0, -d_z, dy\right) \qquad \text{und} \qquad
  f = d \times e \text{.}
\]
Mit der Matrix $M$ werden $d$, $e$ und $f$ auf $x$, $y$ und $z$ abgebildet.
\[
  M =
  \begin{pmatrix}
    d^T \\
    e^T \\
    f^T
  \end{pmatrix}
  =
  \begin{pmatrix}
    d_x & d_y & d_z \\
    e_x & e_y & e_z \\
    f_x & f_y & f_z
  \end{pmatrix}
\]
Im Koordinatensystem $(d, e, f)$ kann dann mit $R_x(\phi)$ rotiert werden.
Mit Rücktransformation ergibt sich
\[
  R_{d,\phi} = M^{-1} \cdot R_x(\phi) \cdot M
\]


\end{document}
