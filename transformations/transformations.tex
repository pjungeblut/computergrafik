\chapter{Transformationen und homogene Koordinaten}

\section{Transformationsgruppen}
\begin{itemize}
  \item \introduce{Euklidische Transformationen} erhalten Abstände und Inhaltsgrößen (Flächeninhalt, Volumen).
  Zu ihnen zählen die Translation, die Identität sowie die Rotation.
  \item \introduce{Ähnlichkeitsabbildungen} erhalten Winkel.
  Alle euklidischen Transformationen sind Ähnlichkeitsabbildungen.
  Außerdem gehören isotrope Skalierungen in diese Klasse.
  \item \introduce{Lineare Abbildungen} enthalten alle Skalierungen, Rotationen,  Spiegelungen und Scherungen.
  Sie sind \introduce{additiv}, $T(p + q) = T(p) + T(q)$, und \introduce{homogen}, $T(\lambda p) = \lambda T(p)$.
  \item \introduce{Affine Abbildungen} enthalten die linearen Abbildungen sowie Translationen.
  Parallele Linien bleiben bei affinen Abbildungen erhalten.
  Weiter sind affine Abbildungen \introduce{teilverhältnistreu}.
  \introduce{Projektive Abbildungen} enthalten alle affinen Abbildungen.
  Die einzige Forderung ist jedoch, dass Geraden auf Geraden abgebildet werden.
\end{itemize}

\section{2D Transformationen}

Transformationen lassen sich durch Matrizen darstellen.
Dies erlaubt eine leichte Hintereinanderausführung.
Dazu müssen nur die einzelnen Transformationsmatrizen multipliziert werden.

\subsection{Skalierung}
Eine \introduce{Skalierung} ändert Längen und Winkel (nur bei nicht isotropen Skalierungen, $s_x \neq s_y$).
\[
  \mathrm{scale}(s_x, s_y) =
  \begin{pmatrix}
    s_x & 0 \\
    0   & s_y
  \end{pmatrix}
\]

\subsection{Scherung (Transvektion)}
Unter einer \introduce{Scherung} versteht man die Verschiebung entlang einer Achse.
Flächeninhalte bleiben daher erhalten.
Es kann entlang beliebiger Achsen geschert werden.
\[
  \mathrm{shear}_x(s) =
  \begin{pmatrix}
    1 & s \\
    0 & 1
  \end{pmatrix}
  \qquad
  \mathrm{shear}_y(s) =
  \begin{pmatrix}
    1 & 0 \\
    s & 1
  \end{pmatrix}
\]

\subsection{Spiegelung}
\introduce{Spiegelungen} an einer Koordinatenachse sind negative Skalierungen.
Es kann entlang beliebiger Achsen gespiegelt werden.
\[
  \mathrm{reflect}_x =
  \begin{pmatrix}
    -1 & 0 \\
    0  & 1
  \end{pmatrix}
  \qquad
  \mathrm{reflect}_y =
  \begin{pmatrix}
    1 & 0 \\
    0 & -1
  \end{pmatrix}
\]

\subsection{Rotation}
Eine \introduce{Rotation} beschreibt eine Drehung um einen Winkel $\phi$.
\[
  \mathrm{rotate}(\phi)
  \begin{pmatrix}
    \cos \phi & -\sin \phi \\
    \sin \phi & \cos \phi
  \end{pmatrix}
\]

\section{3D Transformationen}

\subsection{Rotation}
Die folgenden drei Matrizen drehen um die x-, y- und z-Achse.
\begin{align*}
  \mathrm{R}_x(\phi) &=
  \begin{pmatrix}
    1 & 0         & 0 \\
    0 & \cos \phi & -\sin \phi \\
    0 & \sin \phi & \cos \phi
  \end{pmatrix}
  \qquad
  \mathrm{R}_y(\phi) =
  \begin{pmatrix}
    \cos \phi  & 0 & \sin \phi \\
    0          & 1 & 0 \\
    -\sin \phi & 0 & \cos \phi
  \end{pmatrix} \\
  \mathrm{R}_z(\phi) &=
  \begin{pmatrix}
    \cos \phi  & -\sin \phi & 0 \\
    \sin \phi  & \cos \phi  & 0 \\
    0          & 0          & 1
  \end{pmatrix}
\end{align*}
Allgemein werden Rotationen durch \introduce{orthogonale Matrizen} beschrieben.
Sie sind orientierungserhaltend und ihre Zeilen- bzw. Spaltenvektoren sind paarweise orthonormal.
Eine quadratische, reele Matrix $M$ ist orhtogonal, wenn $M^T \cdot M = M \cdot M^T = I$ gilt.
Es gilt also $M^{-1} = M^T$.

Seien $u$, $v$ und $w$ die Basisvektoren eines orthonormalen Koordinatensystems.
Die Rotationsmatrix
\[
  R_{uvw} = 
  \begin{pmatrix}
    u_x & u_y & u_z \\
    v_x & v_y & v_z \\
    w_x & w_y & w_u
  \end{pmatrix}
\]
bildet die Basisvektoren $u$, $v$ und $w$ auf die kartesischen Achsen ab.
Umgekehrt bildet $R_{uvw}^T$ die kartesischen Einheitsvektoren auf $u$, $v$ und $w$ ab.

Neben den orthogonalen Matrizen können Rotationen auch über sog. \introduce{Euler-Rotationen} berechnet werden.
Dabei wird nacheinander um verschiedene Koordinatenachsen rotiert.
Es gibt zum Beispiel die folgenden drei Möglichkeiten:
\[
  R_z \to R_x \to R_z \qquad
  R_z \to R_y \to R_z \qquad
  R_x \to R_y \to R_z
\]
Sind die Rotationswinkel um die x-, y- und z-Achse $\psi$, $\theta$ und $\phi$, ergibt sich die Rotationsmatrix
\begin{align*}
  R &= R_z(\phi) \cdot R_y(\theta) \cdot R_x(\psi) \\
  &=
  \begin{pmatrix}
    \cos\theta\cos\phi &
    \sin\psi\sin\theta\cos\phi-\cos\psi\sin\phi &
    \cos\psi\sin\theta\cos\phi+\sin\psi\sin\phi \\
    \cos\theta\sin\phi &
    \sin\psi\sin\theta\sin\phi+\cos\psi\cos\phi &
    \cos\psi\sin\theta\sin\phi-\sin\psi\cos\phi \\
    -\sin\theta &
    \sin\psi\cos\theta &
    \cos\psi\cos\theta
  \end{pmatrix}
  \text{.}
\end{align*}
Die Winkel $\psi$, $\theta$ und $\phi$ heißen \introduce{Euler-Winkel} und beschreiben zusammen mit Festlegung der Achsen und der Reihenfolge die Orientierung eines Objekts.

Oft will man um eine Achse $d$ und einen Winkel $\phi$ rotieren.
Dazu wird ein Orthonormalsystem mit $d$ als eine der Achsen benötigt.
Sei $d = \left(d_x, d_y, d_z\right)$ mit $\norm{d} = 1$.
Wähle nun \zB
\[
  e = \frac{1}{\sqrt{d_y^2+ d_z^2}}\left(0, -d_z, dy\right) \qquad \text{und} \qquad
  f = d \times e \text{.}
\]
Mit der Matrix $M$ werden $d$, $e$ und $f$ auf $x$, $y$ und $z$ abgebildet.
\[
  M =
  \begin{pmatrix}
    d^T \\
    e^T \\
    f^T
  \end{pmatrix}
  =
  \begin{pmatrix}
    d_x & d_y & d_z \\
    e_x & e_y & e_z \\
    f_x & f_y & f_z
  \end{pmatrix}
\]
Im Koordinatensystem $(d, e, f)$ kann dann mit $R_x(\phi)$ rotiert werden.
Mit Rücktransformation ergibt sich
\[
  R_{d,\phi} = M^{-1} \cdot R_x(\phi) \cdot M
\]
